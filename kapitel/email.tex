\section{Emailverschlüsselung}

\subsection{Wozu braucht man die überhaupt?}

\begin{frame}{Wie funktioniert Email?}
  \begin{itemize}
    \item im Grunde eine einfache Textdatei
      \begin{itemize}
	\item Metadaten
	\item Inhalt
      \end{itemize}
    \item wird von Server zu Server weitergereicht
    \item und auf dem Server des Empfängers gespeichert
    \item Abruf per Browser oder Mailprogramm
    \item in der Regel bleiben die Mails auf dem Server gespeichert
  \end{itemize}
\end{frame}

\begin{frame}{Wie funktioniert Email?}
  \begin{itemize}
    \item Webmail per Browser ist in der Regel verschlüsselt
      \begin{itemize}
	\item HTTPS
	\item Schlossymbol im Browser
      \end{itemize}
    \item Zugriff per Mailprogramm sollte das auch sein
      \begin{itemize}
	\item „SSL, immer“
	\item „TLS, immer“
      \end{itemize}
    \item Übertragung zwischen den Servern
      \begin{itemize}
	\item leider oft unverschlüsselt
	\item als Nutzer kaum nachprüfbar
      \end{itemize}
  \end{itemize}

  \note{<+FIXXME+>}
\end{frame}

\begin{frame}{Wie funktioniert Email?}
  \begin{itemize}
    \item nicht sicherer als eine Postkarte
    \item Mails liegen unverschlüsselt auf dem Server des Providers
    \item Zugriff möglich durch
      \begin{itemize}
	\item Angestellte
	\item Einbrecher (physisch oder über das Netz)
	\item Strafverfolgungsbehörden
	\item Geheimdienste
	\item … oder einfach durch eine Panne
      \end{itemize}
  \end{itemize}

  \note{<+FIXXME+>}
\end{frame}

\subsection{Wie macht man das?}

\begin{frame}{Was kann ich verschlüsseln?}
  \begin{block}{}
    Verschlüsseln lässt sich nur der \textbf{Inhalt} der Mail, nicht die
    Metadaten!
  \end{block}

  \begin{itemize}
    \item Absender
      \begin{itemize}
	\item Emailadresse
	\item Anschlusskennung (IP-Adresse)
      \end{itemize}
    \item Datum und Uhrzeit
    \item Empfänger
  \end{itemize}

  bleiben immer unverschlüsselt.

  \note{<+FIXXME+>}
\end{frame}

\begin{frame}{OpenPGP}
  \begin{itemize}
    \item OpenPGP ist ein freier Standard
    \item asymetrisches Verschlüsselungsverfahren
    \item ermöglicht:
      \begin{itemize}
	\item Vertraulichkeit (Verschlüsselung)
	  \begin{itemize}
	    \item liest jemand mit?
	  \end{itemize}
	\item Authentisierung (Signaturen)
	  \begin{itemize}
	    \item ist mein Gegenüber der, für den er sich ausgibt?
	  \end{itemize}
      \end{itemize}
  \end{itemize}

  \note{<+FIXXME+>}
\end{frame}

\begin{frame}{OpenPGP}
  jeder braucht 2 zusammengehörige Schlüssel:
  \begin{itemize}
    \item öffentlicher Schlüssel
      \begin{itemize}
	\item wird veröffentlicht oder den Partnern zur Verfügung gestellt
	\item Verschlüsselung
	\item Überprüfung einer Signatur
      \end{itemize}
    \item geheimer Schlüssel
      \begin{itemize}
	\item \textbf{niemals} aus der Hand geben
	\item Entschlüsselung
	\item Signatur
      \end{itemize}
  \end{itemize}

  \note{<+FIXXME+>}
\end{frame}

\begin{frame}{Schlüsselerstellung}
  \begin{itemize}
    \item man braucht nur einen Namen und eine Emailadresse
    \item … und das richtige Programm
    \item öffentlichen Schlüssel bekanntmachen
      \begin{itemize}
	\item selbst zum Download anbieten, z.B. auf der eigenen Homepage
	\item Hochladen auf einen Schlüsselserver
      \end{itemize}
  \end{itemize}

  \uncover<2->{Das ist im Grunde alles, was man braucht. Aber:}

  \note{<+FIXXME+>}
\end{frame}

\begin{frame}{}
  \begin{alertblock}{Achtung}
    \textbf{Jeder} kann zu \textbf{beliebigen} Namen und Emailadressen einen
    Schlüssel erzeugen!
  \end{alertblock}

  Noch leichter lässt sich in einer Email eine gefälschte Absenderadresse
  angeben.

  \note{<+FIXXME+>}
\end{frame}

\begin{frame}{Schlüsselüberprüfung}
  \begin{itemize}
    \item jeder Schlüssel hat einen eindeutigen Fingerabdruck
    \item Austausch des Fingerabdrucks über einen getrennten Kanal
      \begin{itemize}
	\item am besten: persönlich
	\item Telefon
	\item ganz altmodisch per Brief
	\item \textbf{nicht} über das Netz
      \end{itemize}
  \end{itemize}

  \uncover<2->{Was ist, wenn ich eine Email bekomme, den Schlüssel aber noch
  nicht kenne?}

  \note{<+FIXXME+>}
\end{frame}

\begin{frame}{Web of Trust}
  \begin{itemize}
    \item Lösung: Web of Trust („Netz des Vertrauens“)
    \item man „unterschreibt“ überprüfte Schlüssel
    \item „ich habe überprüft und bestätige, daß die Person die ist, für die sie
      sich ausgibt“
    \item hat jemand, den ich überprüft habe, diesen neuen Schlüssel bereits
      geprüft?
    \item funktioniert auch rekursiv
    \item deshalb: \textbf{sorgfältig} prüfen vor dem Unterschreiben eines
      Schlüssels
  \end{itemize}

  \note{<+FIXXME+>}
\end{frame}
