\section{Festplattenverschlüsselung}

\subsection{Wozu braucht man die überhaupt?}

\begin{frame}
  \begin{itemize}
    \item Verlust des Geräts / der Festplatte
      \begin{itemize}
	\item Notebook verloren
	\item Einbruch
      \end{itemize}
    \item alle unverschlüsselten Daten sind sofort zugreifbar
      \begin{itemize}[<+->]
	\item z.B. der OpenPGP-Schlüssel
	\item oder die Partyfotos vom letzten Wochenende
	\item oder die Steuererklärung
	\item … you name it
      \end{itemize}
  \end{itemize}

  \note{<+FIXXME+>}
\end{frame}

\begin{frame}
  \begin{itemize}
    \item Lösung: Verschlüsselung der Festplatte
    \item \textbf{wichtig:} schützt nur zuverlässig, wenn
      \begin{itemize}
	\item das Kennwort („Passphrase“) kompliziert genug ist
	\item das Gerät aus ist, wenn es verloren geht
	\item Standby schützt nicht!
      \end{itemize}
  \end{itemize}

  \note{<+FIXXME+>}
\end{frame}

\subsection{Wie macht man das?}

\begin{frame}{Linux}
  \begin{itemize}
    \item Linux bietet Festplattenverschlüsselung bei der Installation an
      (LUKS)
    \item für einzelne Dateien/Ordner: TrueCrypt
  \end{itemize}

  \note{<+FIXXME+>}
\end{frame}

\begin{frame}{Mac OS}
  \begin{itemize}
    \item Mac OS kann von Haus aus Festplattenverschlüsselung (FileVault)
    \item nachträgliche Verschlüsselung der Festplatte möglich
    \item für einzelne Dateien/Ordner: TrueCrypt
  \end{itemize}

  \note{<+FIXXME+>}
\end{frame}

\begin{frame}{Windows}
  \begin{itemize}
    \item nur manche Varianten können Verschlüsselung von Haus aus (Enterprise,
      Ultimate)
    \item Abhilfe schafft TrueCrypt
    \item nachträgliche Verschlüsselung der Festplatte möglich
  \end{itemize}

  \note{<+FIXXME+>}
\end{frame}

\begin{frame}{}
  \begin{alertblock}{Achtung}
    \centerline{Mac OS und Windows sind inhärent unsicher!}
  \end{alertblock}

  \begin{itemize}
    \item Apple und Microsoft sind amerikanische Unternehmen
    \item Mac OS und Windows sind proprietär und closed source
    \item unter Umständen sind Hintertüren für Geheimdienste eingebaut
  \end{itemize}

  \uncover<2->{\centerline{Wer sicher(er) gehen will, setzt auf ein freies
  Betriebssystem.}}

  \note{<+FIXXME+>}
\end{frame}
